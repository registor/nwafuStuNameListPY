% 使用ctexart文档类(用XeLaTeX编译,直接支持中文)
\documentclass{ctexart}

% 设置页面大小
\usepackage{geometry}
\geometry{% 
  paper=a4paper,
  top=2.5cm,
  bottom=2.1cm,
  left=2.5cm,
  right=2.5cm,
  headheight=0.8cm,
  headsep=0.7cm,
  footskip=1.4cm,
} 

% 载入拼音宏包
\usepackage{xpinyin}
% 设置拼音与正文字号比例
\xpinyinsetup{ratio=0.5}

% 从CSV文件生成表格宏包
\usepackage{pgfplotstable}
% 多行宏包
\usepackage{multirow}

% 数据格式设置
\pgfplotstableset{
  col sep=comma,
  header=true,
  font=\small,
}

% =========================================================
% 从namelist.csv中读入数据
% 该文件第1行是标题后续各行是具体数据
% 前3列分别是学号、姓名、班级,最后一列是备注,建议不要更改标题名称
% 从namelist.csv文件中中数据可以为空,但逗号不能省略
% 如下是namelist.csv文件结构示例:
% stno,name,class,1,2,3,4,5,6,7,8,9,10,11,12,notes
% 2019012800,许涵启,计算机类1905,,,,,,,,,,,,,
% ......
% 2019012830,云睿航,计算机类1905,,,,,,,,,,,,,
% ,,,,,,,,,,,,,,,
% ......
% ,,,,,,,,,,,,,,,
% =========================================================
\pgfplotstableread{namelist.csv}{\loadedtable}

\def\coursename{C语言程序设计}
\def\courseid{1091102}
\def\courseidx{2}
\def\teacher{耿楠}
\begin{document} %在document环境中撰写文档
\pagestyle{empty}
\def\arraystretch{1.2}
\pgfplotstabletypeset[
  % 各列局中,带表格竖线
  column type/.add={|c|}{},% results in ’|c’
  % 表头设计
  every head row/.style={
    before row={
      % \pgfplotstablecols为总列数
      \multicolumn{\pgfplotstablecols}{c}{\bf\large 西北农林科技大学2019秋季学期学生课程考勤表}\\      
      \multicolumn{\pgfplotstablecols}{l}{课程名称:\coursename\hfill 课程代码:\courseid\hfill 课序号:\courseidx\hfill 任课教师:\teacher}\\
      \hline
      % 计算平时考勤登记列数
      \pgfmathtruncatemacro{\len}{\pgfplotstablecols-3}
      % 将平时考勤登记列数存入\colslen宏
      \global\let\colslen\len
      \multirow{2}{*}{学号}&\multirow{2}{*}{姓名}&\multirow{2}{*}{班级}&\multicolumn{\colslen}{c|}{平时考勤}\\
      \cline{4-\pgfplotstablecols}
    }, 
  },
  % 每行后画线
  after row={\hline},
  % 表尾设计
  every last row/.style={
    % 考勤标记
    after row={
      \hline
      \multicolumn{16}{|l|}{备注:出勤:$\surd$\ 缺勤:$\times$\ 迟到:$\bigtriangleup$\ 早退:$\bigcirc$\ 事假:$\bigtriangledown$\ 病假:$\oplus$}\\
      \hline
    },
  },
  % 学号列
  columns/stno/.style={string type,
    column name={}, % 不输出标题名称
    column type=|c,
  },
  % 姓名列
  columns/name/.style={string type,
    column name={},  % 不输出标题名称
    postproc cell content/.style={@cell content=\xpinyin*{##1}}, % 加注拼音
    column type=|c,
  },
  % 班级列
  columns/class/.style={string type,
    column name={},  % 不输出标题名称
    column type=|c,
  },
  % 其它各列采用默认格式
  % 备注列
  columns/notes/.style={
    column name={备注},
    column type=c,
  },  
]{\loadedtable}

% 如果需要多个班级,请准备数据后,用\newpage分页后,复制以上代码排版

\end{document}

%%% Local Variables:
%%% mode: latex
%%% TeX-master: t
%%% End:
